\documentclass[a4paper,10pt]{article}
\usepackage[utf8x]{inputenc}

\usepackage[spanish]{babel}
\selectlanguage{spanish}
\spanishdecimal{.}
\addto\captionsspanish{\def\tablename{Cuadro}}
\usepackage{fancyhdr}
\usepackage{graphics}
\usepackage[dvips]{graphicx}
\usepackage[normal]{caption2}
\usepackage{amsfonts,amssymb,amsmath,amsthm}
\usepackage[T1]{fontenc}
\usepackage{moreverb}

%% Declaracion de comandos %%

\newtheorem{lema}{Lema}
\newtheorem{teor}{Teorema}
\newtheorem{propos}{Proposici\'on}
\newtheorem{corol}{Corolario}

\newcommand{\mivec}[1]{\mathbf{#1}}
\newcommand{\vers}[1]{\mivec{\check{#1}}}
\newcommand{\deriv}[2]{\frac{\mathrm{d}#1}{\mathrm{d}#2}}
\newcommand{\expo}[1]{~10^{#1}}
\newcommand{\uni}[1]{\mathrm{#1}} 

\newcommand{\prop}[1]{\begin{propos} #1 \end{propos}}
\newcommand{\teo}[1]{\begin{teor} #1 \end{teor}}
\newcommand{\cor}[1]{\begin{corol} #1 \end{corol}}
\newcommand{\lem}[1]{\begin{lema} #1 \end{lema}}

%% Encabezado y Pie de Pagina %%

\pagestyle{plain}
\lhead{}
\chead{}
\rhead{}
\cfoot{\thepage}
\renewcommand{\footrulewidth}{0.4pt}

\makeindex

%opening
\title{Computaci\'on Gr\'afica \\Ray Tracer \\Grupo 2}
\author{Guillermo Campelo\\Juan Ignacio Go\~ni\\Juan Tenaillon\\Santiago
V\'azquez}

\begin{document}
\label{introduccion}

\maketitle

\begin{abstract}
Se dise\~n\'o y desarroll\'o un motor de Ray Tracing, determinando
intersecciones entre rayos y objetos, considerando luminosidad, reflexiones,
refracciones, sombras, texturas y anti-aliasing. 
\end{abstract}

\section{Introducci\'on}
\label{intro}

A lo largo de este informe, detallaremos el dise\~no y desarrollo del motor de
Ray Tracing y explicaremos en detalle c\'omo decidimos implementar las distintas
partes que componen a la aplicaci\'on.

Para realizar el mismo, partimos del trabajo pr\'actico anterior, que
consisti\'o en
el desarrollo de un Ray Caster.  A partir del motor de ray casting,
construimos encima
el motor de ray tracing, que toma en consideraci\'on muchas cosas que el ray
caster no ten\'ia.
Entre estas cosas se encuentran las luces, sombras, reflecciones y refracciones,
texturas y la
posibilidad de tener movilidad en la c\'amara entre otras.

Adem\'as de explicar los detalles del dise\~no y de la implementaci\'on,
mostraremos imagenes generadas utilizando el motor haciendo uso de las distintas
opciones que presenta.

Tambi\'en desarrollamos t\'ecnicas de optimizaci\'on para hacer m\'as eficiente
al motor y, como resultado, generar im\'agenes de igual calidad a mayor
velocidad.  En nuestro caso, implementamos Octrees, debido a que nos brinda
tanto vol\'umenes envolventes como tambi\'en subdivisi\'on espacial.  Una
explicaci\'on m\'as detallada de esto podr\'a ser encontrada en la secci\'on
\ref{octree}.

En la secci\'on \ref{escena} detallaremos cada uno de los elementos que componen
a una escena, como las c\'amaras, las luces y los objetos que la pueblan.  En la
siguiente secci\'on, la \ref{motor}, explicaremos m\'as en detalle el motor de
ray tracing, explicando qu\'e es, c\'omo funciona, las caracter\'isticas del
mismo, las optimizaciones que decidimos implementar y una serie de pruebas
realizadas con el motor.
En la secci\'on \ref{uso} explicaremos c\'omo se usa la aplicaci\'on y las
diferentes opciones para ejecutarla.  Por \'ultimo, en la secci\'on \ref{conc},
daremos nuestras conclusiones al respecto sobre el motor que implementamos.

\end{document}
