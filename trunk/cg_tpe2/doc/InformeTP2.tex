\documentclass[a4paper,10pt]{article}
\usepackage[utf8x]{inputenc}

\usepackage[spanish]{babel}
\selectlanguage{spanish}
\spanishdecimal{.}
\addto\captionsspanish{\def\tablename{Cuadro}}
\usepackage{fancyhdr}
\usepackage{graphics}
\usepackage[dvips]{graphicx}
\usepackage[normal]{caption2}
\usepackage{amsfonts,amssymb,amsmath,amsthm}
\usepackage[T1]{fontenc}
\usepackage{moreverb}

%% Declaracion de comandos %%

\newtheorem{lema}{Lema}
\newtheorem{teor}{Teorema}
\newtheorem{propos}{Proposici\'on}
\newtheorem{corol}{Corolario}

\newcommand{\mivec}[1]{\mathbf{#1}}
\newcommand{\vers}[1]{\mivec{\check{#1}}}
\newcommand{\deriv}[2]{\frac{\mathrm{d}#1}{\mathrm{d}#2}}
\newcommand{\expo}[1]{~10^{#1}}
\newcommand{\uni}[1]{\mathrm{#1}} 

\newcommand{\prop}[1]{\begin{propos} #1 \end{propos}}
\newcommand{\teo}[1]{\begin{teor} #1 \end{teor}}
\newcommand{\cor}[1]{\begin{corol} #1 \end{corol}}
\newcommand{\lem}[1]{\begin{lema} #1 \end{lema}}

%% Encabezado y Pie de Pagina %%

\pagestyle{plain}
\lhead{}
\chead{}
\rhead{}
\cfoot{\thepage}
\renewcommand{\footrulewidth}{0.4pt}

\makeindex

%opening
\title{Computaci\'on Gr\'afica \\Ray Tracer \\Grupo 2}
\author{Guillermo Campelo\\Juan Ignacio Go\~ni\\Juan Tenaillon\\Santiago V\'azquez}

\begin{document}
\label{introduccion}

\maketitle

\begin{abstract}
Se dise\~n\'o y desarroll\'o un motor de Ray Tracing, determinando intersecciones entre rayos y objetos, considerando luminosidad, reflexiones, refracciones, sombras, texturas y anti-aliasing. 
\end{abstract}

\section{Introducci\'on}

A lo largo de este informe, se detallar\'a el dise\~no y desarrollo del motor de Ray Tracing.  Explicaremos en detalle c\'omo decidimos implementar las distintas partes que componen a la aplicaci\'on.

Adem\'as de explicar los detalles del dise\~no y de la implementaci\'on, mostraremos imagenes generadas utilizando el motor, haciendo uso de las distintas opciones.

Tambi\'en se desarrollaron m\'etodos de vol\'umenes envolventes y subdivisi\'on espacial para incrementar la eficiencia del motor.  En particular, se desarroll\'o la t\'ecnica de Octrees, que nos brinda tanto vol\'umenes envolventes como subdivisi\'on espacial.


En la Secci\'on \ref{primitivas} explicamos las primitivas implementadas, y como fueron complementadas.  En la Secci\'on \ref{camara} explicamos los dos tipos de c\'amaras implementadas, pinhole y thinlens, luego en la Secci\'on \ref{luces} explicamos las luces implementadas

\end{document}
