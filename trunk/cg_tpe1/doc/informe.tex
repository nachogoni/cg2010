\documentclass[a4paper,10pt]{article}

%% Paquetes Adicionales %%

\usepackage[spanish]{babel}
\selectlanguage{spanish}
\spanishdecimal{.}
\addto\captionsspanish{\def\tablename{Cuadro}}
\usepackage{fancyhdr}
\usepackage{graphics}
\usepackage[dvips]{graphicx}
\usepackage[normal]{caption2}
\usepackage{amsfonts,amssymb,amsmath,amsthm}
\usepackage[T1]{fontenc}
\usepackage{moreverb}

%% Declaracion de comandos %%

\newtheorem{lema}{Lema}
\newtheorem{teor}{Teorema}
\newtheorem{propos}{Proposici\'on}
\newtheorem{corol}{Corolario}

\newcommand{\mivec}[1]{\mathbf{#1}}
\newcommand{\vers}[1]{\mivec{\check{#1}}}
\newcommand{\deriv}[2]{\frac{\mathrm{d}#1}{\mathrm{d}#2}}
\newcommand{\expo}[1]{~10^{#1}}
\newcommand{\uni}[1]{\mathrm{#1}} 

\newcommand{\prop}[1]{\begin{propos} #1 \end{propos}}
\newcommand{\teo}[1]{\begin{teor} #1 \end{teor}}
\newcommand{\cor}[1]{\begin{corol} #1 \end{corol}}
\newcommand{\lem}[1]{\begin{lema} #1 \end{lema}}

%% Encabezado y Pie de Pagina %%

\pagestyle{plain}
\lhead{}
\chead{}
\rhead{}
\cfoot{\thepage}
\renewcommand{\footrulewidth}{0.4pt}

\makeindex

%% Titulo %%
\begin{document}
\title{{Computaci\'on Gr\'afica \\ Trabajo pr\'actico Especial 1 - Ray Caster\\ Grupo 2}}
\author{Guillermo Campelo\\Juan Ignacion Go\~ni\\Juan Tenaillon\\Santiago Vazquez}
\date{}

%% Comienzo del documento %%

\maketitle

\begin{abstract}

Se dise\~no y desarroll\'o un motor de Ray Casting que determine intersecciones entre rayos y objetos permitiendo representar en pantalla un conjunto de escenas predefinidas.

\textbf{Palabras Clave: }\emph{Ray Casting, escena, rayos, objetos.}.
\end{abstract}

%\thispagestyle{fancy}

%% COMIENZO DEL TEXTO %%

\section{Introducci\'on}
\label{introduccion}

En este informe, se explicar\'a el dise\~no y el desarrollo del motor de Ray Casting, explicando cada una de sus componentes y que nos motivo a desarrollarlas de la manera en que lo hicimos.

En la segunda secci\'on nos enfocaremos en el motor de Ray Casting y como se utiliza, luego en la tercera secci\'on nos enfocaremos en la escena y los objetos que la componen, como sean las esferas, 
los triangulos y los cuadrilateros.  En la cuarta secci\'on nos enfocaremos en los resultados obtenidos y por \'ultimo, en la quinta secci\'on, presentamos nuestras conclusiones.
\section{Ray Caster}
\label{raycaster}

\subsection{Descripci\'on}
\label{subseccion}
Ray Casting es una t\'ecnica que utiliza la intersecci\'on de rayos con una superficie para resolver un conjunto de problemas en el campo de la computaci\'on gr\'afica.

 

DEMO TABLA

\begin{table}[h]
\begin{center}
\begin{tabular}{|c|c|c|c|c|c|}
\hline
LARGO & DESTINO & ORIGEN & COMANDO & DATO & CRC \\
\hline
\end{tabular}
\caption{Formato y header del paquete de datos}
\label{formato_paquete_tabla}
\end{center}
\end{table}

DEMO ITEMIZE

\begin{itemize}
	\item{COMANDO:} 0x04
	\item{DATO:} 1 byte con el c\'odigo de error.

	El valor 0x00 indica un error de CRC en el paquete. 
	En este caso, tambi\'en se agrega el paquete con el CRC err\'oneo y luego el CRC esperado.

	El valor 0x01 indica que el comando es desconocido para la placa destinataria.

	Cualquier otro valor, indica un error que depende de la placa y el comando enviado.
\end{itemize}


DEMO IMG

\begin{figure}
\centering
%%\includegraphics[scale=0.75]{daisychain_diagram.png}
\caption{Diagrama general del m\'etodo daisy chain}
\label{daisychain_diagram}
\end{figure}

\end{document}