\documentclass[a4paper,10pt]{article}
\usepackage[utf8x]{inputenc}
\usepackage{fancyhdr}
\usepackage{graphics}
\usepackage{amsmath}    % need for subequations
\usepackage{graphicx}   % need for figures
\usepackage{verbatim}   % useful for program listings
\usepackage{color}      % use if color is used in text
\usepackage{subfigure}  % use for side-by-side figures
\usepackage{hyperref}   % use for hypertext links, including those to external

\pagestyle{fancy}

\lhead[\thepage]{Grupo Número 2}      % Note the different brackets!
\rhead[Grupo Número 2]{Julio 2010}

%opening
\title{Trabajo Práctico Final}
\author{Guillermo Campelo\\Juan Ignacio Goñi\\Juan
Tenaillon\\Santiago Vazquez}

\begin{document}

\maketitle

\begin{abstract}
El siguiente informe describe cómo se diseñó e implementó el Rally Uribe 100K. 
Mediante el uso de imágenes se describe cómo fueron diseñadas e implementadas
las distintas partes que componen al total de la aplicación.  También se
presentan los inconvenientes que surgieron a lo largo del desarrollo y cómo el
equipo decidió solucionarlos.  Por último se explicará cómo se utiliza el juego,
cómo se modifican las distintas opciones de configuración del mismo y las
conclusiones.
\end{abstract}

\section{Introducción}

Haciendo uso del Framework JMonkey, se diseñó y desarrolló un juego de rally,
llamado Rally Uribe 100K.  A continuación, se describe cómo fue realizado el
mismo, qué características posee y la forma en que se utiliza el mismo.

Para el diseño e implementación del juego, partimos de la base del ejemplo
proporcionado por el Framework utilizado, y tomando esa base, se realizó
el desarrollo de las nuevas características.

En la sección \ref{el_juego} se explicará cómo es el juego, los distintos
estados del mismo, cómo es el sistema de puntaje y los sonidos del mismo.

En la sección \ref{caracteristicas} se explicará qué características
adicionales fueron desarrolladas para el juego, como las texturas procedurales,
los screenshots, los arbustos billboard, los skybox intercambiables y demás
características del mismo.

En la sección \ref{mododeuso} se explicará cómo se utiliza el juego, explicando
las teclas a utilizar por defecto, además de explicar cómo utilizar las
distintas opciones del menú.

En la sección \ref{configuracion} se mostrarán las distintas opciones de
configuración, así como los valores recomendados para la misma.  También se
explicará cómo definir nuevas pistas, para que el usuario del mismo pueda crear
sus propios recorridos y utilizarlos con el juego.

Por último, en la sección \ref{conclusiones} se presentarán las conclusiones
alcanzadas por el equipo, además de posibles extensiones al mismo y la
enseñanza obtenida del diseño y desarrollo del mismo.

\section{El Juego}
\label{el_juego}

En esta sección se explicará el juego en sí.  El objetivo del mismo, el modo de
juego, los distintos estados, cómo es el sistema de puntajes y los sonidos
utilizados en el mismo.

\subsection{Descripción}
El juego consiste en recorrer una pista de carreras, ambientada como si fuera
de Rally.  El ambiente en el cual se mueve el auto, está compuesto de un
terreno, con una pista delimitada, con un conjunto de objetos que la componen.

El objetivo del mismo, es dar tres vueltas al recorrido propuesto.  Además de
completar la totalidad de las vueltas, el jugador debe transitar por la
totalidad de los checkpoints, en el orden determinado por la pista.  En caso de
saltearse uno el juego no le permitirá transitar exitosamente por los
siguientes, y el jugador deberá volver hacia atrás y transitar por el
checkpoint adeudado.

Al finalizar las tres vueltas, el jugador recibirá el tiempo de su recorrido, y
mientras más bajo sea ese tiempo, más chances tendrá de ingresar al listado de
los puntajes más altos.
\subsection{Modo de Juego}
Hay un solo modo de juego, y este consiste en intentar realizar el menor tiempo
posible en la realización de las tres vueltas a la pista.
\subsection{Estados del Juego}
Hay tres estados diferenciados en el juego.  Antes de comenzar a jugar, durante
el juego y luego de jugar.

Antes de comenzar a jugar, el jugador encuentra el menú principal de la
aplicación, el cual le presenta todas las opciones disponibles.  Una vez que el
usuario selecciona la opción para comenzar a jugar, se realiza la transición al
siguiente estado, el juego en sí.

\begin{figure}
\begin{minipage}[b]{0.5\linewidth}
\centering
 \includegraphics[scale=0.250]{./main_menu.png}
 % startinggrid.png: 640x480 pixel, 72dpi, 22.58x16.93 cm, bb=0 0 640 480
 \caption{Menú Principal.}
\label{fig:figure0}
\end{minipage}
\hspace{0.5cm}
\begin{minipage}[b]{0.5\linewidth}
\centering
 \includegraphics[scale=0.250]{./options_menu.png}
 % startinggrid.png: 640x480 pixel, 72dpi, 22.58x16.93 cm, bb=0 0 640 480
 \caption{Menú de Opciones.}
\label{fig:figure00}
\end{minipage}
\end{figure}

En este estado, es cuando el jugador maneja el vehículo a través del camino
propuesto.  El vehículo en esta ocasión, puede colisionar contra los distintos
objetos que componen la escena, como los árboles, los límites del terreno y las
pirámides presentes en el recorrido.  Una vez finalizado el recorrido, se
realiza la transición al tercer y último estado del juego.

En este tercer y último estado, al usuario se le presenta el menú nuevamente,
donde podrá ver los puntajes máximos hasta el momento o comenzar nuevamente el
juego para tratar de mejorar su tiempo.

\subsection{Puntajes}
El juego tiene la opción de ver quienes fueron los jugadores que realizaron los
diez mejores tiempo en el recorrido a la pista.

Una vez finalizado el recorrido, se le presentará al jugador su tiempo total de
recorrido, y si el tiempo es lo suficientemente bueno, deberá ingresar su
nombre para comenzar a figurar en el listado de los mejores tiempos.

En el menú inicial del juego, el jugador puede elegir la opción de highscores,
para ver cuales son los mejores tiempos a la pista y quien fue el jugador que
los hizo.
\subsection{Sonidos}
El juego, como todos los de su tipo, tiene diferentes sonidos que ayudan a
ambientar el mismo.

Comenzando por los sonidos de ambiente, donde se decidió que se iban a utilizar
dos pistas para tener cierto cambio en la música de ambiente pero sin entrar en
la exageración de utilizar muchas pistas.

También fueron utilizados dos sonidos para ser utilizados por el vehículo para
denotar situaciones en las que se está acelerando y situaciones en las que no.

Por último, fue utilizado un sonido para los momentos en que el vehículo
colisiona contra algun objeto presente en el terreno.

Todos los sonidos propios del vehículo y su entorno fueron obtenidos de
Internet, de páginas de uso libre y gratuito.  Los temas utilizados para la
música ambiental, corresponden a la autoría de Santiago Vazquez, integrante del
equipo.

\begin{figure}
\begin{minipage}[b]{0.5\linewidth}
\centering
 \includegraphics[scale=0.250]{./startinggrid.png}
 % startinggrid.png: 640x480 pixel, 72dpi, 22.58x16.93 cm, bb=0 0 640 480
 \caption{Grilla de Partida del Rally Uribe 100K. Bajo nivel de detalle.}
\label{fig:figure1}
\end{minipage}
\hspace{0.5cm}
\begin{minipage}[b]{0.5\linewidth}
\centering
 \includegraphics[scale=0.250]{./startinggrid_high.png}
 % startinggrid.png: 640x480 pixel, 72dpi, 22.58x16.93 cm, bb=0 0 640 480
 \caption{Grilla de Partida del Rally Uribe 100K. Alto nivel de detalle.}
\label{fig:figure2}
\end{minipage}
\end{figure}


\section{Características Extra}
\label{caracteristicas}

Dentro de las características opcionales a implementar, el equipo se inclinó
por las siguientes: texturas procedurales, capturas de pantallas, árboles con
billboards \textbf{PONER EL RESTO DE LAS CARACTERISTICAS EXTRA}

\subsection{Texturas Procedurales}

Para la realización de las texturas procedurales, el desarrollo fue basado en
el diseñado para la extensión del \textit{Trabajo Práctico Número 2 - Ray
Tracer}.  Se implementaron dos texturas procedurales, piedra y marmol.

\begin{figure}
\begin{minipage}[b]{0.5\linewidth}
\centering
 \includegraphics[scale=0.250]{./marble.png}
 % startinggrid.png: 640x480 pixel, 72dpi, 22.58x16.93 cm, bb=0 0 640 480
 \caption{Textura Procedural: Marmol.}
\label{fig:figure5}
\end{minipage}
\hspace{0.5cm}
\begin{minipage}[b]{0.5\linewidth}
\centering
 \includegraphics[scale=0.250]{./stone.png}
 % startinggrid.png: 640x480 pixel, 72dpi, 22.58x16.93 cm, bb=0 0 640 480
 \caption{Textura Procedural: Piedra}
\label{fig:figure6}
\end{minipage}
\end{figure}

Donde en la figura \ref{fig:figure5} y \ref{fig:figure6} se puede observar las
dos texturas procedurales realizadas, marmol y piedra respectivamente.

Las dos fueron realizadas utilizando Perlin Noise, realizando las sumas
sucesivas de los ruidos para distintas frecuencias.  Para más detalle,
remitirse al informe del \textit{Trabajo Práctico Número 2 - Extensión}.

\subsection{Capturas de Pantalla}

Otra de las características desarrolladas, fue la posibilidad de obtener
capturas de pantallas de cualquier momento del juego.  Para realizar dicha
acción, hay que presionar la tecla $0$ (cero) en cualquier momento del mismo.

Las imágenes producidas, son almacenadas en la raíz del proyecto, y el nombre
con el que se guardan es la fecha en que fueron realizados.

Esta característica fue muy utilizada a la hora de realizar este informe.

\subsection{Billboards}

Otra característica implementada, fue el hecho de contar con árboles realizados
con texturas de tipo billboard.

En las imágenes presentadas con antelación y más precisamente en la figura
\ref{fig:figure7}, puede observarse cómo este tipo de árboles son utilizados a
lo largo de todo el recorrido; donde también cabe mencionar que el vehículo
puede colisionar contra este tipo de texturas.

Algo interesante sobre este tipo de texturas es que nos brindan la posibilidad
de verlas correctamente desde cualquier ángulo, además permitiendo que se vean
con claridad las otras texturas detras de ellas.

\begin{figure}
 \centering
 \includegraphics[scale=0.4]{./billboard.png}
 % billboard.png: 640x480 pixel, 72dpi, 22.58x16.93 cm, bb=0 0 640 480
 \caption{Árboles con Texturas Billboard.}
 \label{fig:figure7}
\end{figure}


\subsection{Agregar características opcionales desarrolladas}

\section{Modo de Uso}
\label{mododeuso}
En esta sección, comentaremos la forma de jugar al Rally Uribe 100K.  Cabe
destacar que estas instrucciones están basadas en la configuración default del
juego, ya que cabe la posibilidad de modificar las teclas a utilizar.

\subsection{¿Cómo se Juega?}

\subsection{Los Menúes}
Para movernos a través de los menúes, es necesario utilizar las flechas hacia
arriba, abajo, izquierda, derecha y el enter.  Con las flechas de arriba y
abajo, marcaremos la opción deseada, mientras que con la tecla de enter,
seleccionaremos la misma.

En las opciones del menú que contengan distintas posibilidades de
configuración, ya sea el momento del día en cual vamos a correr, estado del
sonido y demás opciones, es necesario utilizar las flechas hacia la derecha e
izquierda para cambiar la opción.
\section{Configuración}
\label{configuracion}

\subsection{Configuración del Juego}

\subsection{Configuración de Nuevas Pistas}

\section{Conclusión}
\label{conclusiones}

\end{document}
